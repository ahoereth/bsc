\addtocontents{toc}{\protect\vspace{2\beforebibskip}}
\pagestyle{plain}


\chapter*{Abkürzungsverzeichnis}
\label{sec:abbreviations}
\addcontentsline{toc}{chapter}{\tocEntry{Abkürzungsverzeichnis}}

\begin{acronym}[UMLX]
  \acro{MVP}{Minimum Viable Product}
  \acro{SaaS}{Software-as-a-Service}
  \acro{SPA}{Single Page Application}
  \acro{BRAK}{Bundesrechtsanwaltskammer}
  \acro{beA}{besonderes elektronisches Anwaltspostfach}
  \acro{ES}{ECMAscript}
  \acro{MVC}{Model-View-Controller}
  \acro{REST}{Representational State Transfer}
  \acro{API}{Application Programming Interface}
  \acro{JIT}{Just-in-time-Kompilierung}
  \acro{AJAX}{asynchronous JavaScript}
  \acro{PWA}{Progressive Web Application}
  \acro{EGVP}{Elektronisches Gerichts- und Verwaltungspostfach}
  \acro{BMJV}{Bundesministerium der Justiz und für Verbraucherschutz}
  \acro{BGBL}{Bundesgesetzblatt}
  \acro{HTTP}{Hypertext Transfer Protocol}
  \acro{HTTPS}{Hypertext Transfer Protocol Secure}
  \acro{AWS}{Amazon Web Services}
  \acro{EC2}{Elastic Cloud Computing}
  \acro{RDS}{Relational Database Service}
  \acro{JWT}{JSON Web Token}
  \acro{CA}{Certificate Authority}
  \acro{AMP}{Accelerated Mobile Pages}
  \acro{SSR}{Server Side Rendering}
  \acro{DOM}{Document Object Model}
  \acro{CSS}{Cascading Style Sheets}
\end{acronym}



% Allow flexible page breaks in the reference list.
\newcommand\orghypertarget{}
\let\orghypertarget\hypertarget
\renewcommand\hypertarget[2]{\leavevmode\orghypertarget{#1}{#2}}


\chapter*{Referenzen}
\addcontentsline{toc}{chapter}{\tocEntry{Referenzen}}

